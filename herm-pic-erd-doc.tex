%% herm-pic-erd-doc.tex
%% Copyright 2003-2005 S. Schoradt
%
% This work may be distributed and/or modified under the
% conditions of the LaTeX Project Public License, either version 1.3
% of this license or (at your option) any later version.
% The latest version of this license is in
%   http://www.latex-project.org/lppl.txt
% and version 1.3 or later is part of all distributions of LaTeX
% version 2003/12/01 or later.
%
% This work has the LPPL maintenance status "maintained".
% 
% This Current Maintainer of this work is S. Schoradt <sven@dex.de>
%
% This work consists of the files listed in Manifest.txt.
%
\documentclass[a4paper,11pt]{article}

\setlength{\parindent}{0pt}
\setlength{\parskip}{5pt plus 2pt minus 1pt}
\frenchspacing
\sloppy

\setlength{\unitlength}{.3cm}

\usepackage[erd]{herm-pic}

\setlength{\hermunit}{.6cm}

\pagestyle{headings}

\author{Sven Schoradt \\ (sven@dex.de)}

\title{herm-pic - Graphical extension package for LaTeX2e to produce 
       HER-Diagramms\\The Entity-Relationship Diagramm mode}

\begin{document}

\maketitle

\section{General remarks}

For a wider view on HermPic see the HermPic documentation.

\section{The schema elements}

All of the macros from the normal herm-pic you find also in the ERD mode.
Some, especialy \verb|\cluster| are not known in ERD diagramms. In the 
ERD mode there are also some special macros that you not find outside this mode.

So herm-pic schemas are compatible to the ERD mode, but ERD diagramms work not 
in the normal herm-pic environment.

On entities and relations there are no changes, but there are some new 
object types.

That are weak entity and weak relation types, that work like the coresponding 
strong types with the macros \verb|\wentity| and \verb|\wrelation|.

\begin{verbatim}
\begin{schema}(20,14)
  \wentity(1,11){E1}
  \entity(15,11){E2}
  \entity(15,1){E3}
  \entity(1,1){E4}
  \wrelation(8,6){R}
  \conn(R,E1){(5,1000)}
  \conn(R,E2){(1,n)}
  \conn(R,E3){(n,m)}
  \conn(R,E4){(1,1)}
\end{schema}
\end{verbatim}

produces

\begin{schema}(20,14)
  \wentity(1,11){E1}
  \entity(15,11){E2}
  \entity(15,1){E3}
  \entity(1,1){E4}
  \wrelation(8,6){R}
  \conn(R,E1){(5,1000)}
  \conn(R,E2){(1,n)}
  \conn(R,E3){(n,m)}
  \conn(R,E4){(1,1)}
\end{schema}

And there is also the is-a type. 

\subsection{Is A}

To use the is-a concept in your schema, you can use the \verb|\isa| macro.
The syntax is as followed:

\begin{verbatim}
\isa(x,y){name}
\end{verbatim}

This draws an is-a sign with the height of 2 and a width of 4 units.
The name is only for internal use and have to be unique.

\paragraph{Example:}

\begin{verbatim}
\begin{schema}(16,4)
\isa(6,1){isa1}
\end{schema}
\end{verbatim}

produces this

\begin{schema}(16,4)
\isa(6,1){isa1}
\end{schema}

\subsection{Attributes}

Attribute are drawn as an oval with the name in it conected with the object.

\paragraph{Example:}

\begin{verbatim}
\begin{schema}(10,6)
\entity(4,2){Person}
\attr[lm]{Person}{left}
\attr[om]{Person}{over}
\attr[rm]{Person}{right}
\attr[um]{Person}{under}
\end{schema}
\end{verbatim}

produces

\begin{schema}(10,6)
\entity(4,2){Person}
\attr[lm]{Person}{left}
\attr[om]{Person}{over}
\attr[rm]{Person}{right}
\attr[um]{Person}{under}
\end{schema}

and

\begin{schema}(10,4)
\entity(1,1){Person}
\attr[rm]{Person}{\key{key}}
\end{schema}


\begin{schema}(14,20)
  \entity[Person](5,13){person}
  \relation[arbeitet](5,3){works}
  \attr[ol]{Person}{ol}
  \attr[om]{Person}{om}
  \attr[or]{Person}{or}
  \attr[ro]{Person}{ro}
  \attr[rm]{Person}{rm}
  \attr[ru]{Person}{ru}
  \attr[ur]{Person}{ur}
  \attr[um]{Person}{um}
  \attr[ul]{Person}{ul}
  \attr[lu]{Person}{lu}
  \attr[lm]{Person}{lm}
  \attr[lo]{Person}{lo}
  \attr[ol]{arbeitet}{ol}
  \attr[om]{arbeitet}{om}
  \attr[or]{arbeitet}{or}
  \attr[ro]{arbeitet}{ro}
  \attr[rm]{arbeitet}{rm}
  \attr[ru]{arbeitet}{ru}
  \attr[ur]{arbeitet}{ur}
  \attr[um]{arbeitet}{um}
  \attr[ul]{arbeitet}{ul}
  \attr[lu]{arbeitet}{lu}
  \attr[lm]{arbeitet}{lm}
  \attr[lo]{arbeitet}{lo}
\end{schema}

\subsection{Special attribute types}

Special attribute types are the multivalued attribute ( \verb|\multivaluedattr|),
the optional attribute (\verb|\optattr|) and derived attributes (\verb|\derivedattr|).

There are also structured attributes called \verb|\complexattr|. 
The name of a complex attribute should not consist of special letters.

A structured attribute can consit of normal attributes, multivalued attributes 
and derived attributes.

They was used like simple attributes but there is no star or plus version available.

\paragraph{Example:}

\begin{verbatim}
\begin{schema}(14,30)
  \entity[entity](5,23){Entity}
  \attr[lm]{entity}{a11}
  \multivaluedattr[lo]{entity}{a12}
  \optattr[ol]{entity}{a1}
  \derivedattr[om]{entity}{a2}
  \complexattr[um]{entity}{cattr}
  \attr[ro]{cattr}{a8}
  \derivedattr[ru]{cattr}{a9}
  \multivaluedattr[lo]{cattr}{a10}
  \complexattr[um]{cattr}{cattr2}
  \attr[um]{cattr2}{attrr}
\end{schema}
\end{verbatim}

produces

\begin{schema}(14,14)
  \entity[entity](5,8){Entity}
  \attr[lm]{entity}{a11}
  \multivaluedattr[lo]{entity}{a12}
  \optattr[ol]{entity}{a1}
  \derivedattr[om]{entity}{a2}
  \complexattr[um]{entity}{cattr}
  \attr[ro]{cattr}{a8}
  \derivedattr[ru]{cattr}{a9}
  \multivaluedattr[lo]{cattr}{a10}
  \complexattr[um]{cattr}{cattr2}
  \attr[um]{cattr2}{attrr}
\end{schema}

\section{Connections between schema elements}

Connections are drawn without arrows in ER diagramms. Only if you use a 
connection to or from an is-a type, ther will be drawn arrows.

\paragraph{Example:}

\begin{verbatim}
\begin{schema}(20,5)
  \entity(1,1){Person}
  \entity(13,1){Team}
  \relation[arbeitetin](7,1){arbeitet\_in}
  \connection(arbeitetin,Person){(0,3)}
  \connection(arbeitetin,Team1){(2,10)}
\end{schema}
\end{verbatim}

produces

\begin{schema}(20,5)
  \entity(1,1){Person}
  \entity(13,1){Team}
  \relation[arbeitetin](7,1){arbeitet\_in}
  \connection(arbeitetin,Person){(0,3)}
  \connection(arbeitetin,Team){(2,10)}
\end{schema}

and 

\begin{verbatim}
\begin{schema}(20,5)
  \entity(1,1){Person}
  \entity(13,1){Team}
  \relation*(7,1){arbeitet\_in}
  \connection*(7,2)(5,2){(0,3)}
  \connection*(11,2)(13,2){(2,10)}
\end{schema}
\end{verbatim}

produces this

\begin{schema}(20,4)
  \entity(1,1){Person}
  \entity(13,1){Team}
  \relation*(7,1){arbeitet\_in}
  \connection*(7,2)(5,2){(0,3)}
  \connection*(11,2)(13,2){(2,10)}
\end{schema}

and

\begin{verbatim}
\begin{schema}(20,14)
  \entity(1,11){Unter1}
  \entity(15,11){Unter2}
  \isa(8,6){isa1}
  \entity(8,1){Ober}
  \conn(Unter1,isa1){}
  \conn(Unter2,isa1){}
  \conn(isa1,Ober){}
\end{schema}
\end{verbatim}

produces

\begin{schema}(20,14)
  \entity(1,1){subtype1}
  \entity(15,1){subtype2}
  \isa(8,6){isa1}
  \entity(8,11){type}
  \conn(subtype1,isa1){}
  \conn(subtype2,isa1){}
  \conn(isa1,type){}
\end{schema}

\section{Specialising}

To implement ER specialising you can use the macros

\begin{verbatim}
  \dspecial(x,y){iname}
  \dspecial*(x,y){iname}
  \ospecial(x,y){iname}
  \ospecial*(x,y){iname}
 
  \sconn(oname1,oname2)
\end{verbatim}

The \verb|\dspecial| macro represents an disjunct specialisation, the 
\verb|\ospecial| macro an overlapping specialisation. The star versions 
represents the total versions of the specialisation.

The \verb|\sconn| macro represents the specilisation between two objects. 
The objects must give in the order subtype, supertype.

This works like this:

\begin{verbatim}
\begin{schema}(14,30)
  \entity[entity7](5,26){Entity}
  \entity[entity8](1,22){Entity}
  \entity[entity9](9,22){Entity}
  \sconn(entity8,entity7)
  \sconn(entity9,entity7)
  \entity[entity](5,27){Entity}
  \dspecial(6,23){test1}
  \connection(entity,test1){}
  \entity[entity2](1,19){Entity}
  \entity[entity3](9,19){Entity}
  \sconn(entity2,test1)
  \sconn(entity3,test1)
  \entity[entity4](5,16){Entity}
  \ospecial*(6,12){test2}
  \connection(entity4,test2){}
  \entity[entity5](1,8){Entity}
  \entity[entity6](9,8){Entity}
  \sconn(entity5,test2)
  \sconn(entity6,test2)
\end{schema}
\end{verbatim}

\begin{schema}(14,28)
  \entity[entity7](5,26){Entity}
  \entity[entity8](1,22){Entity}
  \entity[entity9](9,22){Entity}
  \sconn(entity8,entity7)
  \sconn(entity9,entity7)
  \entity[entity](5,19){Entity}
  \dspecial(6,15){test1}
  \connection(entity,test1){}
  \entity[entity2](1,11){Entity}
  \entity[entity3](9,11){Entity}
  \sconn(entity2,test1)
  \sconn(entity3,test1)
  \entity[entity4](5,8){Entity}
  \ospecial*(6,4){test2}
  \connection(entity4,test2){}
  \entity[entity5](1,0){Entity}
  \entity[entity6](9,0){Entity}
  \sconn(entity5,test2)
  \sconn(entity6,test2)
\end{schema}

\section{A more or less complex example}

This more or less complex example shows you the usage of all the macros together.

\begin{verbatim}
\setlength{\hermunit}{.5cm}
\begin{schema}(30,17)
\entity(15,8){Dokument}
\attr[lo]{Dokument}{titel}
\attr[ro]{Dokument}{gueltig\_ab}
\attr[lu]{Dokument}{datei}
\attr[ru]{Dokument}{gueltig\_bis}
%
\entity(6,12){Satzung}
\attr[om]{Satzung}{\key{snummer}}
%
\entity(24,12){Kategorie}
\attr[rm]{Kategorie}{\key{kname}}
%
\entity(0,4){Amt}
\attr[rm]{Amt}{\key{aname}}
%
\entity(15,0){Wort}
\attr[ro]{Wort}{\key{wort}}
\attr[ru]{Wort}{soundex}
%
\relation(18,12){in}
\conn(in,Dokument){}
\conn(in,Kategorie){}
%
\relation(12,12){zu}
\conn(zu,Dokument){(1,1)}
\conn(zu,Satzung){}
%
\relation(24,8){unter}
\conn*(24,9)(24,12){}
\conn*(28,9)(28,12){}
%
\relation(0,0){uebergeordnet}
\conn*(0,1)(0,4){(0,1)}
\conn*(4,1)(4,4){}
%
\relation(0,12){verantwortlich}
\conn(verantwortlich,Satzung){(1,1)}
\conn(verantwortlich,Amt){}
%
\relation(12,4){Schlagwort}
\conn(Schlagwort,Dokument){}
\conn(Schlagwort,Wort){}
%
\relation(18,4){Inhalt}
\attr[rm]{Inhalt}{anzahl}
\conn(Inhalt,Dokument){}
\conn(Inhalt,Wort){}
\end{schema}
\end{verbatim}

produces the schema

\setlength{\hermunit}{.5cm}
\begin{schema}(30,17)
\entity(15,8){Dokument}
\attr[lo]{Dokument}{titel}
\attr[ro]{Dokument}{gueltig\_ab}
\attr[lu]{Dokument}{datei}
\attr[ru]{Dokument}{gueltig\_bis}
%
\entity(6,12){Satzung}
\attr[om]{Satzung}{\key{snummer}}
%
\entity(24,12){Kategorie}
\attr[rm]{Kategorie}{\key{kname}}
%
\entity(0,4){Amt}
\attr[rm]{Amt}{\key{aname}}
%
\entity(15,0){Wort}
\attr[ro]{Wort}{\key{wort}}
\attr[ru]{Wort}{soundex}
%
\relation(18,12){in}
\conn(in,Dokument){}
\conn(in,Kategorie){}
%
\relation(12,12){zu}
\conn(zu,Dokument){(1,1)}
\conn(zu,Satzung){}
%
\relation(24,8){unter}
\conn*(24,9)(24,12){}
\conn*(28,9)(28,12){}
%
\relation(0,0){uebergeordnet}
\conn*(0,1)(0,4){(0,1)}
\conn*(4,1)(4,4){}
%
\relation(0,12){verantwortlich}
\conn(verantwortlich,Satzung){(1,1)}
\conn(verantwortlich,Amt){}
%
\relation(12,4){Schlagwort}
\conn(Schlagwort,Dokument){}
\conn(Schlagwort,Wort){}
%
\relation(18,4){Inhalt}
\attr[rm]{Inhalt}{anzahl}
\conn(Inhalt,Dokument){}
\conn(Inhalt,Wort){}
\end{schema}


\end{document}
