\documentclass[a4paper,11pt]{article}

\setlength{\parindent}{0pt}
\setlength{\parskip}{5pt plus 2pt minus 1pt}
\frenchspacing
\sloppy

%\tracingmacros=2
%\tracingcommands=1
%\tracingonline=1

\usepackage[
            verbatim,
            center,
            box,
            erd
            ]{herm-pic}

\setlength{\hermunit}{.6cm}

\author{Sven Schoradt \\ (sven@dex.de)}

\title{herm-pic - Graphical extension package for LaTeX2e to produce 
       HER-Diagramms\\The Entity-Relationship Diagramm mode examples}

\begin{document}

\maketitle

\begin{abstract} 
This file holds some examples for the ER diagramm mode of the herm-pic latex package.
\end{abstract}

\section{Attributes}

Here you see the way how attributes are visualized.

\begin{schema}(14,30)
  \entity[entity](5,23){Entity}
  \relation[relation](5,13){Relation}
  \attr[ol]{entity}{\key{a1}}
  \attr[om]{entity}{\key{a2}}
  \attr[or]{entity}{\key{a3}}
  \attr[ro]{entity}{\key{a4}}
  \attr[rm]{entity}{\key{a5}}
  \attr[ru]{entity}{\key{a6}}
  \attr[ur]{entity}{a7}
  \attr[um]{entity}{a8}
  \attr[ul]{entity}{a9}
  \attr[lu]{entity}{a10}
  \attr[lm]{entity}{a11}
  \attr[lo]{entity}{a12}
  \attr[ol]{relation}{a1}
  \attr[om]{relation}{a2}
  \attr[or]{relation}{a3}
  \attr[ro]{relation}{a4}
  \attr[rm]{relation}{a5}
  \attr[ru]{relation}{a6}
  \attr[ur]{relation}{a7}
  \attr[um]{relation}{a8}
  \attr[ul]{relation}{a9}
  \attr[lu]{relation}{a10}
  \attr[lm]{relation}{a11}
  \attr[lo]{relation}{a12}
\end{schema}

\section{Connection examples}

The way connections are made.

\begin{schema}(20,14)
  \entity(1,11){E1}
  \entity(15,11){E2}
  \entity(15,1){E3}
  \entity(1,1){E4}
  \relation(8,6){R}
  \connection(R,E1){(5,1000)}
  \connection(R,E2){(1,n)}
  \connection(R,E3){(n,m)}
  \connection(R,E4){(1,1)}
\end{schema}

\begin{schema}(20,14)
  \entity(1,11){E1}
  \entity(8,11){E2}
  \entity(15,11){E3}
  \entity(15,6){E4}
  \entity(15,1){E5}
  \entity(8,1){E6}
  \entity(1,1){E7}
  \entity(1,6){E8}
  \relation(8,6){R}
  \connection(R,E1){}
  \connection(R,E2){}
  \connection(R,E3){}
  \connection(R,E4){}
  \connection(R,E5){}
  \connection(R,E6){}
  \connection(R,E7){}
  \connection(R,E8){}
\end{schema}

\begin{schema}(20,14)
  \entity(1,11){E1}
  \entity(15,11){E2}
  \entity(15,1){E3}
  \entity(1,1){E4}
  \relation(8,6){R}
  \conn(R,E1){(5,1000)}
  \conn(R,E2){(1,n)}
  \conn(R,E3){(n,m)}
  \conn(R,E4){(1,1)}
\end{schema}

\begin{schema}(20,14)
  \entity(1,11){E1}
  \entity(8,11){E2}
  \entity(15,11){E3}
  \entity(15,6){E4}
  \entity(15,1){E5}
  \entity(8,1){E6}
  \entity(1,1){E7}
  \entity(1,6){E8}
  \relation(8,6){R}
  \conn(R,E1){}
  \conn(R,E2){}
  \conn(R,E3){}
  \conn(R,E4){}
  \conn(R,E5){}
  \conn(R,E6){}
  \conn(R,E7){}
  \conn(R,E8){}
\end{schema}

\section{ERD extensions}

Some extensions that are made for the erd stuff.

\begin{schema}(20,14)
  \entity(1,11){under1}
  \entity(15,11){under2}
  \isa(8,6){isa1}
  \entity(8,1){over}
  \conn(under1,isa1){}
  \conn(under2,isa1){}
  \conn(isa1,over){}
\end{schema}

\begin{schema}(20,14)
  \wentity(1,11){E1}
  \entity(15,11){E2}
  \entity(15,1){E3}
  \entity(1,1){E4}
  \wrelation(8,6){R}
  \conn(R,E1){(5,1000)}
  \conn(R,E2){(1,n)}
  \conn(R,E3){(n,m)}
  \conn(R,E4){(1,1)}
\end{schema}

\begin{schema}(14,30)
  \entity[entity](5,23){Entity}
  \attr[lm]{entity}{a11}
  \multivaluedattr[lo]{entity}{a12}
  \optattr[ol]{entity}{a1}
  \derivedattr[om]{entity}{a2}
  \complexattr[or]{entity}{cattr1}
\end{schema}

\subsection{complex attributes}

\begin{schema}(14,35)
  \entity[entity](5,28){Entity}
  \relation[relation](5,18){Relation}
  \complexattr[ol]{entity}{ca1}
  \complexattr[om]{entity}{ca2}
  \complexattr[or]{entity}{ca3}
  \complexattr[ro]{entity}{ca4}
  \complexattr[rm]{entity}{ca5}
  \complexattr[ru]{entity}{ca6}
  \complexattr[ur]{entity}{ca7}
  \complexattr[um]{entity}{ca8}
  \complexattr[ul]{entity}{ca9}
  \complexattr[lu]{entity}{ca10}
  \complexattr[lm]{entity}{ca11}
  \complexattr[lo]{entity}{ca12}
  \complexattr[ol]{relation}{c1a1}
  \complexattr[om]{relation}{c1a2}
  \complexattr[or]{relation}{c1a3}
  \complexattr[ro]{relation}{c1a4}
  \complexattr[rm]{relation}{c1a5}
  \complexattr[ru]{relation}{c1a6}
  \complexattr[ur]{relation}{c1a7}
  \complexattr[um]{relation}{c1a8}
  \complexattr[ul]{relation}{c1a9}
  \complexattr[lu]{relation}{c1a10}
  \complexattr[lm]{relation}{c1a11}
  \complexattr[lo]{relation}{c1a12}
  \entity[entity2](5,13){Entity}
  \complexattr[um]{entity2}{cluster2}
  \attr[ro]{cluster2}{a8}
  \derivedattr[ru]{cluster2}{a9}
  \multivaluedattr[lo]{cluster2}{a10}
  \complexattr[um]{cluster2}{c2a8}
  \attr[um]{c2a8}{attrr}
\end{schema}

\subsection{specialising}

\begin{schema}(14,30)
  \entity[entity](5,27){Entity}
  \dspecial(6,23){test1}
  \connection(entity,test1){}
  \entity[entity2](1,19){Entity}
  \entity[entity3](9,19){Entity}
  \sconn(entity2,test1)
  \sconn(entity3,test1)
  \entity[entity4](5,16){Entity}
  \ospecial*(6,12){test2}
  \connection(entity4,test2){}
  \entity[entity5](1,8){Entity}
  \entity[entity6](9,8){Entity}
  \sconn(entity5,test2)
  \sconn(entity6,test2)
\end{schema}



\section{Larger example schema}

\setlength{\hermunit}{.5cm}
\begin{schema}(30,15)
\entity(15,8){document}
\attr[lo]{document}{title}
\attr[ro]{document}{valid\_from}
\attr[lu]{document}{file}
\attr[ru]{document}{valid\_untill}
%
\entity(6,12){rule}
\attr[om]{rule}{\key{id}}
%
\entity(24,12){categorie}
\attr[rm]{categorie}{\key{name}}
%
\entity(0,4){organisation}
\attr[rm]{organisation}{\key{name}}
%
\entity(15,0){word}
\attr[ro]{word}{\key{word}}
\attr[ru]{word}{soundex}
%
\relation(18,12){in}
\connection(in,document){}
\connection(in,categorie){}
%
\relation(12,12){to}
\connection(to,document){(1,1)}
\connection(to,rule){}
%
\relation(24,8){sub}
\connection*(24,9)(24,12){}
\connection*(28,9)(28,12){}
%
\relation(0,0){superior}
\connection*(0,1)(0,4){(0,1)}
\connection*(4,1)(4,4){}
%
\relation(0,12){responsible}
\connection(responsible,rule){(1,1)}
\connection(responsible,organisation){}
%
\relation(12,4){catchword}
\connection(catchword,document){}
\connection(catchword,word){}
%
\relation(18,4){content}
\attr[rm]{content}{count}
\connection(content,document){}
\connection(content,word){}
\end{schema}

\end{document}
