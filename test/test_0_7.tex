\documentclass[a4paper,11pt]{article}

\setlength{\parindent}{0pt}
\setlength{\parskip}{5pt plus 2pt minus 1pt}
\frenchspacing
\sloppy

%\tracingmacros=2
%\tracingcommands=1
%\tracingonline=1

%\usepackage[verbatim,center]{herm-pic}
\usepackage[center,box,verbatim]{herm-pic}

\setlength{\hermunit}{.6cm}

\begin{document}

\section{Introduction}

This file is a test case and also an example file for all features of
the {\tt herm-pic} LaTeX package that are added befor version 0.7.0.

You use the {\tt herm-pic} Package in Version \HPlongrevision{} from the \HPdate.

\section{Objects}

Here you see the main objects, provided by herm-pic. This are entities, relations
and clusters.

\setlength{\unitlength}{.6cm}
\begin{schema}(16,14)
  \entity(1,3){Entity}
  \entity*(6,3){Entity\_2}
  \entity[entity3](11,3){Entity\_3}
  \relation(1,6){Relation}
  \relation*(6,6){Relation\_2}
  \relation[relation3](11,6){Relation\_3}
  \cluster(1,9){Cluster}
  \cluster*(6,9){Cluster\_2}
  \cluster[cluster3](11,9){Cluster\_3}
\end{schema}

\section{Attributes}

In this section you see, how attributes can be connected to objects.

\begin{schema}(20,8)
  \entity[entity](2,3){Entity}
  \relation[relation](10,3){Relation}
  \attr[ol]{entity}{\key{a1}}
  \attr[om]{entity}{\key{a2}}
  \attr[or]{entity}{\key{a3}}
  \attr[ro]{entity}{\key{ag4}}
  \attr[rm]{entity}{\key{aA5}}
  \attr[ru]{entity}{\key{aA6}}
  \attr[ur]{entity}{a7}
  \attr[um]{entity}{a8}
  \attr[ul]{entity}{a9}
  \attr[lu]{entity}{a10}
  \attr[lm]{entity}{a11}
  \attr[lo]{entity}{a12}
  \attr[ol]{relation}{a1}
  \attr[om]{relation}{a2}
  \attr[or]{relation}{a3}
  \attr[ro]{relation}{a4}
  \attr[rm]{relation}{a5}
  \attr[ru]{relation}{a6}
  \attr[ur]{relation}{a7}
  \attr[um]{relation}{a8}
  \attr[ul]{relation}{a9}
  \attr[lu]{relation}{a10}
  \attr[lm]{relation}{a11}
  \attr[lo]{relation}{a12}
  \cluster[cluster](18,3){Cluster}
  \attr[ol]{cluster}{a1}
  \attr[om]{cluster}{a2}
  \attr[or]{cluster}{a3}
  \attr[ro]{cluster}{a4}
  \attr[rm]{cluster}{a5}
  \attr[ru]{cluster}{a6}
  \attr[ur]{cluster}{a7}
  \attr[um]{cluster}{a8}
  \attr[ul]{cluster}{a9}
  \attr[lu]{cluster}{a10}
  \attr[lm]{cluster}{a11}
  \attr[lo]{cluster}{a12}
\end{schema}

\section{Connection examples}

And here you can see how objects will be connected.

\begin{schema}(20,14)
  \entity(1,11){E1}
  \entity(15,11){E2}
  \entity(15,1){E3}
  \entity(1,1){E4}
  \relation(8,6){R}
  \connection(R,E1){(5,1000)}
  \connection(R,E2){(1,n)}
  \connection(R,E3){(n,m)}
  \connection(R,E4){(1,1)}
\end{schema}

\begin{schema}(20,14)
  \entity(1,11){E1}
  \entity(8,11){E2}
  \entity(15,11){E3}
  \entity(15,6){E4}
  \entity(15,1){E5}
  \entity(8,1){E6}
  \entity(1,1){E7}
  \entity(1,6){E8}
  \relation(8,6){R}
  \connection(R,E1){}
  \connection(R,E2){}
  \connection(R,E3){}
  \connection(R,E4){}
  \connection(R,E5){}
  \connection(R,E6){}
  \connection(R,E7){}
  \connection(R,E8){}
\end{schema}

\section{Cluster example}

Now also the main use case for clusters.

\begin{schema}(18,10)
  \entity(1,4){person}
  \attr[ol]{person}{name}
  \attr[or]{person}{\key{ssn}}
  \relation(13,7){professor}
  \relation(13,4){assistant}
  \relation(13,1){student}
  \cnamepos{um}
  \cluster(7,4){X}
  \cnamepos{or}
  \attr[ol]{X}{\key{mnr}}
  \connection(X,person){}
  \connection(professor,X){}
  \connection(assistant,X){}
  \connection(student,X){}
\end{schema}

\begin{schema}(22,10)
  \cluster(10,4){cluster}
  \attr[ol]{cluster}{a1}
  \attr[om]{cluster}{a2}
  \attr[or]{cluster}{a3}
  \attr[ro]{cluster}{a4}
  \attr[rm]{cluster}{a5}
  \attr[ru]{cluster}{a6}
  \attr[ur]{cluster}{a7}
  \attr[um]{cluster}{a8}
  \attr[ul]{cluster}{a9}
  \attr[lu]{cluster}{a10}
  \attr[lm]{cluster}{a11}
  \attr[lo]{cluster}{a12}
\end{schema}

\section{A larger example}

And now a larger example. This will be in every example to check that 
no errors are build in between version cahnge.

\setlength{\hermunit}{.5cm}
\begin{schema}(30,15)
\entity(15,8){document}
\attr[lo]{document}{title}
\attr[ro]{document}{valid\_from}
\attr[lu]{document}{file}
\attr[ru]{document}{valid\_untill}
%
\entity(6,12){rule}
\attr[om]{rule}{\key{id}}
%
\entity(24,12){categorie}
\attr[rm]{categorie}{\key{name}}
%
\entity(0,4){organisation}
\attr[rm]{organisation}{\key{name}}
%
\entity(15,0){word}
\attr[ro]{word}{\key{word}}
\attr[ru]{word}{soundex}
%
\relation(18,12){in}
\connection(in,document){}
\connection(in,categorie){}
%
\relation(12,12){to}
\connection(to,document){(1,1)}
\connection(to,rule){}
%
\relation(24,8){sub}
\connection*(24,9)(24,12){}
\connection*(28,9)(28,12){}
%
\relation(0,0){superior}
\connection*(0,1)(0,4){(0,1)}
\connection*(4,1)(4,4){}
%
\relation(0,12){responsible}
\connection(responsible,rule){(1,1)}
\connection(responsible,organisation){}
%
\relation(12,4){catchword}
\connection(catchword,document){}
\connection(catchword,word){}
%
\relation(18,4){content}
\attr[rm]{content}{count}
\connection(content,document){}
\connection(content,word){}
\end{schema}

\subsection{Cluster name positioning}

\begin{schema}(6,40)
\cnamepos{ol}
\cluster*(2,38){Cluster\_ol}
\cnamepos{or}
\cluster*(2,35){Cluster\_or}
\cnamepos{ro}
\cluster*(2,32){Cluster\_ro}
\cnamepos{ru}
\cluster*(2,29){Cluster\_ru}
\cnamepos{ur}
\cluster*(2,26){Cluster\_ur}
\cnamepos{ul}
\cluster*(2,23){Cluster\_ul}
\cnamepos{lu}
\cluster*(2,20){Cluster\_lu}
\cnamepos{lo}
\cluster*(2,17){Cluster\_lo}
\cnamepos{om}
\cluster*(2,14){Cluster\_om}
\cnamepos{rm}
\cluster*(2,11){Cluster\_rm}
\cnamepos{um}
\cluster*(2,8){Cluster\_um}
\cnamepos{lm}
\cluster*(2,5){Cluster\_lm}
\cnamepos{xx}
\cluster*(2,2){Cluster\_xx}
\end{schema}

\end{document}
