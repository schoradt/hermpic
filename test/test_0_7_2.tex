\documentclass[a4paper,11pt]{article}

\setlength{\parindent}{0pt}
\setlength{\parskip}{5pt plus 2pt minus 1pt}
\frenchspacing
\sloppy

%\tracingmacros=2
%\tracingcommands=1
%\tracingonline=1

%\usepackage[verbatim,center]{herm-pic}
\usepackage[center,box,verbatim]{herm-pic}

\setlength{\hermunit}{.6cm}

\begin{document}


\section{Connection examples}

\begin{schema}(20,14)
  \entity(1,11){E1}
  \entity(15,11){E2}
  \entity(15,1){E3}
  \entity(1,1){E4}
  \relation(8,6){R}
  \connection(R,E1){(5,1000)}
  \connection(R,E2){(1,n)}
  \connection(R,E3){(n,m)}
  \connection(R,E4){(1,1)}
\end{schema}

\begin{schema}(20,14)
  \entity(1,11){E1}
  \entity(8,11){E2}
  \entity(15,11){E3}
  \entity(15,6){E4}
  \entity(15,1){E5}
  \entity(8,1){E6}
  \entity(1,1){E7}
  \entity(1,6){E8}
  \relation(8,6){R}
  \connection(R,E1){}
  \connection(R,E2){}
  \connection(R,E3){}
  \connection(R,E4){}
  \connection(R,E5){}
  \connection(R,E6){}
  \connection(R,E7){}
  \connection(R,E8){}
\end{schema}

\begin{schema}(20,14)
  \entity(1,11){E1}
  \entity(15,11){E2}
  \entity(15,1){E3}
  \entity(1,1){E4}
  \relation(8,6){R}
  \conn(R,E1){(5,1000)}
  \conn(R,E2){(1,n)}
  \conn(R,E3){(n,m)}
  \conn(R,E4){(1,1)}
\end{schema}

\begin{schema}(20,14)
  \entity(1,11){E1}
  \entity(8,11){E2}
  \entity(15,11){E3}
  \entity(15,6){E4}
  \entity(15,1){E5}
  \entity(8,1){E6}
  \entity(1,1){E7}
  \entity(1,6){E8}
  \relation(8,6){R}
  \conn(R,E1){}
  \conn(R,E2){}
  \conn(R,E3){}
  \conn(R,E4){}
  \conn(R,E5){}
  \conn(R,E6){}
  \conn(R,E7){}
  \conn(R,E8){}
\end{schema}

\section{Larger example schema}

\setlength{\hermunit}{.5cm}
\begin{schema}(30,15)
\entity(15,8){Dokument}
\attr[lo]{Dokument}{titel}
\attr[ro]{Dokument}{gueltig\_ab}
\attr[lu]{Dokument}{datei}
\attr[ru]{Dokument}{gueltig\_bis}
%
\entity(6,12){Satzung}
\attr[om]{Satzung}{\key{snummer}}
%
\entity(24,12){Kategorie}
\attr[rm]{Kategorie}{\key{kname}}
%
\entity(0,4){Amt}
\attr[rm]{Amt}{\key{aname}}
%
\entity(15,0){Wort}
\attr[ro]{Wort}{\key{wort}}
\attr[ru]{Wort}{soundex}
%
\relation(18,12){in}
\conn(in,Dokument){}
\conn(in,Kategorie){}
%
\relation(12,12){zu}
\conn(zu,Dokument){(1,1)}
\conn(zu,Satzung){}
%
\relation(24,8){unter}
\conn*(24,9)(24,12){}
\conn*(28,9)(28,12){}
%
\relation(0,0){uebergeordnet}
\conn*(0,1)(0,4){(0,1)}
\conn*(4,1)(4,4){}
%
\relation(0,12){verantwortlich}
\conn(verantwortlich,Satzung){(1,1)}
\conn(verantwortlich,Amt){}
%
\relation(12,4){Schlagwort}
\conn(Schlagwort,Dokument){}
\conn(Schlagwort,Wort){}
%
\relation(18,4){Inhalt}
\attr[rm]{Inhalt}{anzahl}
\conn(Inhalt,Dokument){}
\conn(Inhalt,Wort){}
\end{schema}

\end{document}