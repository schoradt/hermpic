\documentclass[a4paper,11pt]{article}

\setlength{\parindent}{0pt}
\setlength{\parskip}{5pt plus 2pt minus 1pt}
\frenchspacing
\sloppy

\setlength{\unitlength}{.3cm}

\usepackage{herm-pic}

\setlength{\hermunit}{.6cm}

\pagestyle{headings}

\author{Sven Schoradt \\ (Sven.Schoradt@informatik.tu-cottbus.de)}

\title{herm-pic - Graphical extension package for LaTeX2e to produce HER-Diagramms}

\begin{document}

\maketitle

\section{General remarks}

The {\tt herm-pic} package, helps you to draw Higher Entity Relationship Diagramms.
This diagramms are the visualisation of HER schemas, an extended database modeling 
concept. For more information to this see the internet presenation of the
database and information system chair of the BTU Cottbus (Germany)
at http://dbis.informatik.tu-cottbus.de/.

\section{Usage}

To use the package you only have to insert the line

\begin{verbatim}
\usepackage{herm-pic}
\end{verbatim}

into the header part of your document. There are no known problems with other
packages or classes. If you find something in this manner, mail to the package 
author.

The package defines a length for the unit length in the schema environments, it was called
\verb|\hermunit|. So you can use pictures and schemas with different unit length in your 
document without redefining the unit length every time. You can set it with
\begin{verbatim}
\setlength{\hermunit}{1cm}
\end{verbatim}
The standard for \verb|\hermunit| is 0.5cm for one unit.

\section{The schema environment}

In the {\tt herm-pic} package I defined the {\tt schema} environment.
This environment is the standard way to use the schema elements, but you 
can also use the schema elements in an ordinary picture environment.

You use the schema environment in the following manner:

\begin{verbatim}
\begin{schema}(x,y)

\end{schema}
\end{verbatim}

where {\tt x} is the width and {\tt y} is the height of 
the schema graphic.

\section{The schema elements}

\subsection{Entities}

To use an entity in your schema, you can use the \verb|\entity| macro.
The syntax is the following:

\begin{verbatim}
\entity(x,y){text}
\end{verbatim}

where {\tt (x,y)} are the coordinates of the left upper edge of the entity
symbol. The {\tt text} was printed as the entity name.

The entity symbol will be a rectangle with the width of 4 and the height of 2.

\paragraph{Example:}
\begin{verbatim}
\begin{schema}(6,4)
\entity(1,1){Person}
\end{schema}
\end{verbatim}

produces this

\begin{schema}(6,4)
\entity(1,1){Person}
\end{schema}

\subsection{Relationships}

To draw relationships in your diagrams, you can use the \verb|\relation| macro.
The syntax is the following:

\begin{verbatim}
\relation(x,y){text}
\end{verbatim}

where {\tt (x,y)} are the coordinates of the left edge. The {\tt text} will be displayed as 
the name of the relationship.

\paragraph{Example:}

\begin{verbatim}
\begin{schema}(6,4)
\relation(1,1){is a}
\end{schema}
\end{verbatim}

produces this

\begin{schema}(6,4)
\relation(1,1){is a}
\end{schema}

\subsection{Attributes}

Attributes in HERM will be a string connected with the entity or relation.
To draw attributes with herm-pic you have to use the macro \verb|\attr| with the syntax:

\begin{verbatim}
\attr(x,y)[pos]{text}
\end{verbatim}

with {\tt (x,y)} is the docking position on the entity or relationship. With the
position you will say the macro in which direction the attribute will docked on
the object. Possible values are {\tt r, l, o, u}, where r stands for right of the object,
l stands for left of the object, o stands for over the object and u stands for under the object.

\paragraph{Example:}

\begin{verbatim}
\begin{schema}(10,6)
\entity(4,2){Person}
\attr(4,3)[l]{left}
\attr(6,4)[o]{over}
\attr(8,3)[r]{right}
\attr(6,2)[u]{under}
\end{schema}
\end{verbatim}

produces this

\begin{schema}(10,6)
\entity(4,2){Person}
\attr(4,3)[l]{left}
\attr(6,4)[o]{over}
\attr(8,3)[r]{right}
\attr(6,2)[u]{under}
\end{schema}

To show that an attribute or an sub-attribute is a key concept for the object, entity or relationship,
you can use the \verb|\key| macro, like this:

\begin{verbatim}
\begin{schema}(10,4)
\entity(1,1){Person}
\attr(4,3)[r]{\key{key}}
\end{schema}
\end{verbatim}

produces this

\begin{schema}(10,4)
\entity(1,1){Person}
\attr(5,2)[r]{\key{key}}
\end{schema}

\subsection{Cluster}

To use the cluster concept in your schema, you can use the \verb|\cluster| macro.
The syntax is as followed:

\begin{verbatim}
\cluster(x,y)
\end{verbatim}

This draws the cluster item, thats a circle with the radius 2 whit a plus in it, on the position 
{\tt (x,y)}.

\paragraph{Example:}

\begin{verbatim}
\begin{schema}(4,4)
\cluster(1,1)
\end{schema}

\end{verbatim}

produces this

\begin{schema}(4,4)
\cluster(1,1)
\end{schema}

\section{Connections between schema elements}

A connection between two schema elements is a vector with an optional
text marking for the cardinality constraint.

The package provides for this the \verb|\connection| macro, with the syntax

\begin{verbatim}
\connection(x1,y1)(x2,y2){mathematical equation}
\end{verbatim}

It draws a vector from the point (x1,y1) to the next possible point in the near
of (x2,y2) and puts the mathematical equation to an adequate point at the end of the 
vector enclosed by \$-signs. 

\paragraph{Example:}

\begin{verbatim}
\begin{schema}(20,5)
  \entity(1,1){Person}
  \entity(13,1){Team}
  \relation(7,1){arbeitet\_in}
  \connection(7,2)(5,2){(0,3)}
  \connection(11,2)(13,2){(2,10)}
\end{schema}
\end{verbatim}

produces this

\begin{schema}(20,4)
  \entity(1,1){Person}
  \entity(13,1){Team}
  \relation(7,1){arbeitet\_in}
  \connection(7,2)(5,2){(0,3)}
  \connection(11,2)(13,2){(2,10)}
\end{schema}

\section{ToDo}

\subsection{Short term development}

\begin{itemize}
\item develop an easier user interface, whith name lookup for entities and
  relations ( comes in version 0.6 )
\item test the package for bugs and for problems whith other packages
\end{itemize}

\subsection{Long term development}

\begin{itemize}
\item improve the drawing for entities and relations (make them as wide as
  needed to display the name inside the object)
\item improve attribute placement
\item improve connection drawing and make it configurable
\item implement an object placement algorithm for authomatic diagramm design
\end{itemize}

\end{document}
