\documentclass[a4paper,11pt]{article}

\setlength{\parindent}{0pt}
\setlength{\parskip}{5pt plus 2pt minus 1pt}
\frenchspacing
\sloppy

\setlength{\unitlength}{.3cm}

\usepackage{herm-pic}

\setlength{\hermunit}{.6cm}

\pagestyle{headings}

\author{Sven Schoradt \\ (Sven.Schoradt@informatik.tu-cottbus.de)}

\title{herm-pic - Graphical extension package for LaTeX2e to produce HER-Diagramms}

\begin{document}

\maketitle

\section{General remarks}

The {\tt herm-pic} package, helps you to draw Higher Entity Relationship Diagramms.
This diagramms are the visualisation of HER schemas, a database modeling 
concept. For more information to this see the original paper at 
http://citeseer.nj.nec.com/thalheim91foundation.html.

\section{Package information}

You use the {\tt herm-pic} Package in Version \HPlongrevision{} from the \HPdate.

This information and some more about the package state you can request from the 
revision macros defined in {\tt herm-rev.sty}. This are

\begin{description}
\item[$\backslash{}$HPdescription] prints a short description of the package
\item[$\backslash{}$HPrevision] prints the full revision information, consisting major, minor and releaseversion
  and a releasecomment like {\tt '-rc2'}
\item[$\backslash{}$HPlongrevision] prints also the version information and the cvs revision number of the 
  {\tt herm-rev.sty} file for consistency checks
\item[$\backslash{}$HPdate] prints the date the revision is last changed
\end{description}

\section{Usage}

To use the package you only have to insert the line

\begin{verbatim}
\usepackage{herm-pic}
\end{verbatim}

into the header part of your document. There are no known problems with other
packages or classes. If you find something in this manner, mail to the package 
author.

\subsection{Package options}

For compatibility to lower versions (up to 0.6.1) there is the option
rev-0-6. This option lets load a minimal pathed version of the last offical 
release of the hermpic package of the 0.6 version tree. 
I think it will be removed in the first release candidate of the next
minor release step (first release candidate of version 0.8 or higher).

{\bf So port your schemas {\tt :-) } }

The package accepts the options {\tt verbatim}, {\tt center} and {\tt box}.

\begin{description}
\item[verbatim] prints some processing information to the log file, its mainly 
  usable for debugging purposes
\item[center] center the scema verticaly on the page
\item[box] draw a box with the exact dimensions of the schema around the schema, so you
  can see if some elements will be drawn outside the schema boarders
\end{description}

\section{The schema environment}

In the {\tt herm-pic} package I defined the {\tt schema} environment.

This environment is the standard way to use the schema elements, but you 
can also use the schema elements in an ordinary picture environment.

You use the schema environment in the following manner:

\begin{verbatim}
\begin{schema}(x,y)

\end{schema}
\end{verbatim}

where {\tt x} is the width and {\tt y} is the height of 
the schema.

To scale the schema you can use the \verb|\hermpic| length.
You set it with

\begin{verbatim}
\setlength{\hermunit}{.6cm}
\end{verbatim}

\section{The schema elements}

To most of the schema element macros there exists a star version that
provides an interface with to point placing and with unlimited names.
For the nonstar version you can give a valid internal name as an optional 
parameter.

The names of object, that are used for internal things should not consist any
active tex stuff, that means no macros, no escaped chars (like the undersgore)
and the on. To garantee that there are no limits to the printed object names 
you can give optional an internal name without these things.

\subsection{Entities}

To use an entity in your schema, you can use the \verb|\entity| macro.
The syntax is the following:

\begin{verbatim}
\entity(x,y){text}
\end{verbatim}

,

\begin{verbatim}
\entity[internal_name](x,y){text}
\end{verbatim}

or 

\begin{verbatim}
\entity*(x,y){text}
\end{verbatim}

where {\tt (x,y)} are the coordinates of the left lower edge of the entity
symbol. The {\tt text} was printed as the entity name. In the first version 
the entity name should only consits of Characters and Numbers, 
but no tex special chars are allowed.
In the other version there is no limitation to the text.

The internal name should only consits of characters and numbers.

The entity symbol will be a rectangle with the width of 4 and the height of 2 
units.

\paragraph{Example:}
\begin{verbatim}
\begin{schema}(17,4)
\entity(1,1){Person}
\entity[person2](7,1){Person\_2}
\entity*(12,1){Person\_3}
\end{schema}
\end{verbatim}

produces this

\begin{schema}(18,4)
\entity(1,1){Person}
\entity[person2](7,1){Person\_2}
\entity*(13,1){Person\_3}
\end{schema}

\subsection{Relationships}

To draw relationships in your diagrams, you can use the \verb|\relation| macro.
The syntax is the following:

\begin{verbatim}
\relation(x,y){text}
\end{verbatim}

,

\begin{verbatim}
\relation[internal_name](x,y){text}
\end{verbatim}

or

\begin{verbatim}
\relation*(x,y){text}
\end{verbatim}

where {\tt (x,y)} are the coordinates of the left edge. The {\tt text} will be displayed as 
the name of the relationship. There are the same restrictions to the name 
how they are descriped for entitys.

\paragraph{Example:}

\begin{verbatim}
\begin{schema}(17,4)
\relation(1,1){Relation}
\relation[relation2](7,1){Relation\_2}
\relation*(12,1){Relation\_3}
\end{schema}
\end{verbatim}

produces this

\begin{schema}(18,4)
\relation(1,1){Relation}
\relation[relation2](7,1){Relation\_2}
\relation*(13,1){Relation\_3}
\end{schema}

\subsection{Cluster}

To use the cluster concept in your schema, you can use the \verb|\cluster| macro.
The syntax is as followed:

\begin{verbatim}
\cluster(x,y){name}
\end{verbatim}

,

\begin{verbatim}
\cluster[internal_name](x,y){name}
\end{verbatim}

or

\begin{verbatim}
\cluster*(x,y){name}
\end{verbatim}

This draws the cluster item, thats a circle with the radius 2 whith a 
plus in it, on the position {\tt (x,y)} and places the name right over 
the cluster.

\paragraph{Example:}

\begin{verbatim}
\begin{schema}(16,4)
\cluster(1,1){}
\cluster(5,1){Cluster 1}
\cluster[cluster2](9,1){Cluster\_2}
\cluster*(13,1){Cluster\_3}
\end{schema}
\end{verbatim}

produces this

\begin{schema}(16,4)
\cluster(1,1){}
\cluster(5,1){Cluster 1}
\cluster[cluster2](9,1){Cluster\_2}
\cluster*(13,1){Cluster\_3}
\end{schema}

To slect the position where the cluster name is displayed, you can use the 
\verb|\cnamepos| macro. The possible values are the same like at the attribute 
possitioning (see there).

\paragraph{Example:}

\begin{verbatim}
\begin{schema}(16,4)
\cnamepos{ru}
\cluster(1,1){}
\cluster(5,1){Cluster 1}
\cluster[cluster2](9,1){Cluster\_2}
\cluster*(13,1){Cluster\_3}
\end{schema}
\end{verbatim}

produces this

\begin{schema}(16,4)
\cnamepos{ru}
\cluster(1,1){}
\cluster(5,1){Cluster 1}
\cluster[cluster2](9,1){Cluster\_2}
\cluster*(13,1){Cluster\_3}
\end{schema}

\subsection{Attributes}

Attributes in HERM will be a string connected with the entity or relation.
To draw attributes with herm-pic you have to use the macro \verb|\attr| with the syntax:

\begin{verbatim}
\attr[pos]{objectname}{attributename}
\end{verbatim}

or

\begin{verbatim}
\attr*[pos](x,y){attributename}
\end{verbatim}

In the non-star version you have only 12 docking points to the object, setable with the pos attribute.
Figure \ref{fig:attr_pos} shows the positionings.

\begin{figure}[htb]
  \centering
  \begin{schema}(20,6)
  \entity(1,2){Person}
  \relation(9,2){arbeitet}
  \attr[ol]{Person}{ol}
  \attr[om]{Person}{om}
  \attr[or]{Person}{or}
  \attr[ro]{Person}{ro}
  \attr[rm]{Person}{rm}
  \attr[ru]{Person}{ru}
  \attr[ur]{Person}{ur}
  \attr[um]{Person}{um}
  \attr[ul]{Person}{ul}
  \attr[lu]{Person}{lu}
  \attr[lm]{Person}{lm}
  \attr[lo]{Person}{lo}
  \attr[ol]{arbeitet}{ol}
  \attr[om]{arbeitet}{om}
  \attr[or]{arbeitet}{or}
  \attr[ro]{arbeitet}{ro}
  \attr[rm]{arbeitet}{rm}
  \attr[ru]{arbeitet}{ru}
  \attr[ur]{arbeitet}{ur}
  \attr[um]{arbeitet}{um}
  \attr[ul]{arbeitet}{ul}
  \attr[lu]{arbeitet}{lu}
  \attr[lm]{arbeitet}{lm}
  \attr[lo]{arbeitet}{lo}
  \cluster(17,2){cluster}
  \attr[ol]{cluster}{ol}
  \attr[om]{cluster}{om}
  \attr[or]{cluster}{or}
  \attr[ro]{cluster}{ro}
  \attr[rm]{cluster}{rm}
  \attr[ru]{cluster}{ru}
  \attr[ur]{cluster}{ur}
  \attr[um]{cluster}{um}
  \attr[ul]{cluster}{ul}
  \attr[lu]{cluster}{lu}
  \attr[lm]{cluster}{lm}
  \attr[lo]{cluster}{lo}
\end{schema}

  \caption{Positioning posibilities for attributes}
  \label{fig:attr_pos}
\end{figure}

in the star version you can set the docking position with the (x,y) coordinates
and the alignment to the object with the pos attribute (l,r,o,u).

\paragraph{Example:}

\begin{verbatim}
\begin{schema}(10,6)
\entity(4,2){Person}
\attr[lm]{Person}{left}
\attr[om]{Person}{over}
\attr[rm]{Person}{right}
\attr[um]{Person}{under}
\end{schema}
\end{verbatim}

produces

\begin{schema}(10,6)
\entity(4,2){Person}
\attr[lm]{Person}{left}
\attr[om]{Person}{over}
\attr[rm]{Person}{right}
\attr[um]{Person}{under}
\end{schema}

To show that an attribute or an sub-attribute is a key concept for the object, entity or relationship,
you can use the \verb|\key| macro, like this:

\begin{verbatim}
\begin{schema}(10,4)
\entity(1,1){Person}
\attr*(4,3)[r]{\key{key}}
\end{schema}
\end{verbatim}

produces this

\begin{schema}(10,4)
\entity(1,1){Person}
\attr[rm]{Person}{\key{key}}
\end{schema}

\section{Connections between schema elements}

A connection between two schema elements is a vector with an optional
text marking for the cardinality constraint.

The package provides for this the \verb|\connection| macro, with the syntax

\begin{verbatim}
\connection(objectname1,objectname2){mathematical equation}
\end{verbatim}

or

\begin{verbatim}
\connection*(x1,y1)(x2,y2){mathematical equation}
\end{verbatim}

It draws a vector from one object to an other. The star version draws 
a vector from the point (x1,y1) to (x2,y2) and puts the mathematical 
equation to an adequate point at the end of the vector enclosed by \$-signs. 

\paragraph{Example:}

\begin{verbatim}
\begin{schema}(20,5)
  \entity(1,1){Person}
  \entity(13,1){Team}
  \relation[arbeitetin](7,1){arbeitet\_in}
  \connection(arbeitetin,Person){(0,3)}
  \connection(arbeitetin,Team1){(2,10)}
\end{schema}
\end{verbatim}

produces

\begin{schema}(20,5)
  \entity(1,1){Person}
  \entity(13,1){Team}
  \relation[arbeitetin](7,1){arbeitet\_in}
  \connection(arbeitetin,Person){(0,3)}
  \connection(arbeitetin,Team){(2,10)}
\end{schema}

, 

\begin{verbatim}
\begin{schema}(15,10)
  \entity(1,4){Person}
  \attr[ol]{Person}{Name}
  \attr[or]{Person}{\key{PNr}}
  \relation(13,7){Professor}
  \relation(13,4){WiMi}
  \relation(13,1){Student}
  \cluster(7,4){Mitarbeiter}
  \attr[ol]{Mitarbeiter}{\key{mnr}}
  \connection(Mitarbeiter,Person){}
  \connection(Professor,Mitarbeiter){}
  \connection(WiMi,Mitarbeiter){}
  \connection(Student,Mitarbeiter){}
\end{schema}
\end{verbatim}

produces

\begin{schema}(15,10)
  \entity(1,4){Person}
  \attr[ol]{Person}{Name}
  \attr[or]{Person}{\key{PNr}}
  \relation(13,7){Professor}
  \relation(13,4){WiMi}
  \relation(13,1){Student}
  \cluster(7,4){Mitarbeiter}
  \attr[ol]{Mitarbeiter}{\key{mnr}}
  \connection(Mitarbeiter,Person){}
  \connection(Professor,Mitarbeiter){}
  \connection(WiMi,Mitarbeiter){}
  \connection(Student,Mitarbeiter){}
\end{schema}

and 

\begin{verbatim}
\begin{schema}(20,5)
  \entity(1,1){Person}
  \entity(13,1){Team}
  \relation*(7,1){arbeitet\_in}
  \connection*(7,2)(5,2){(0,3)}
  \connection*(11,2)(13,2){(2,10)}
\end{schema}
\end{verbatim}

produces this

\begin{schema}(20,4)
  \entity(1,1){Person}
  \entity(13,1){Team}
  \relation*(7,1){arbeitet\_in}
  \connection*(7,2)(5,2){(0,3)}
  \connection*(11,2)(13,2){(2,10)}
\end{schema}

\end{document}
